\documentclass[12pt,a4paper,utf8x]{report}
\usepackage [english]{babel}

% Pour pouvoir utiliser 
\usepackage{ucs}
\usepackage[utf8x]{inputenc}

\usepackage{url} % Pour avoir de belles url
\usepackage {geometry}

% Pour mettre du code source
\usepackage {listings}
% Pour pouvoir passer en paysage
\usepackage{lscape}

% Pour pouvoir faire plusieurs colonnes
\usepackage {multicol}
% Pour crééer un index
\usepackage{makeidx}
\makeindex

% Pour gérer les liens interractifs et les signets Acrobat
\usepackage{hyperref}
\hypersetup{
pdftitle={titre de mon document},
pdfauthor={DEJEAN Charles, POTTIER Vincent},
pdfsubject={Sujet du document},
pdfkeywords={les mots clefs},
bookmarks, % Création du signet
pdfstartview=FitH, % Page de la largeur de la fenêtre
colorlinks=true, % Liens en couleur
linkcolor=black, 	
anchorcolor=black, 	
citecolor=black, 	
filecolor=black, 	
menucolor=black,
runcolor=black,
urlcolor=black, 	
frenchlinks=black,
bookmarksnumbered=true, % Signet numéroté
pdfpagemode=UseOutlines, % Montre les bookmarks.
bookmarksopen =true,
}

% Pour afficher la bibliographie, mais pas nottoc (Table of Contents), notlof (List of Figures) ni notlot (List of Tables)
\usepackage[notlof, notlot]{tocbibind}


% Pour les entetes de page
% \usepackage{fancyheadings}
%\pagestyle{fancy}
%\renewcommand{\sectionmark}[1]{\markboth{#1}{}} 
%\renewcommand{\subsectionmark}[1]{\markright{#1}} 

% Pour l'interligne de 1.5
\usepackage {setspace}
% Pour les marges de la page
\geometry{a4paper, top=2.5cm, bottom=3.5cm, left=1.5cm, right=1.5cm, marginparwidth=1.2cm}

\parskip=5pt %% distance entre § (paragraphe)
\sloppy %% respecter toujours la marge de droite 

% Pour les pénalités :
\interfootnotelinepenalty=150 %note de bas de page
\widowpenalty=150 %% veuves et orphelines
\clubpenalty=150 

%Pour la longueur de l'indentation des paragraphes
\setlength{\parindent}{15mm}

%%%% debut macro pour enlever le nom chapitre %%%%
\makeatletter
\def\@makechapterhead#1{%
  \vspace*{50\p@}%
  {\parindent \z@ \raggedright \normalfont
    \interlinepenalty\@M
    \ifnum \c@secnumdepth >\m@ne
        \Huge\bfseries \thechapter\quad
    \fi
    \Huge \bfseries #1\par\nobreak
    \vskip 40\p@
  }}

\def\@makeschapterhead#1{%
  \vspace*{50\p@}%
  {\parindent \z@ \raggedright
    \normalfont
    \interlinepenalty\@M
    \Huge \bfseries  #1\par\nobreak
    \vskip 40\p@
  }}
\makeatother
%%%% fin macro %%%%

%Couverture 
\title
{
	\normalsize{Master ALMA Nantes\\
	Université de Nantes\\
	2009-2010}\\
	\vspace{15mm}
	\Huge{Project Report for Technique de Développement}
}
\author{DEJEAN Charles, POTTIER Vincent\\
	\vspace{45mm}
}



\begin{document}

\maketitle

%\input{Remerciements}
%\clearpage

\tableofcontents
\clearpage

% Pour avoir un interligne de 1,5
\begin{onehalfspace}

\chapter{Conception}

\section{}

	\subsection{Mise en place}

		Dans un premier temps nous avons cherché à découper le projet en sous parties distinct. Etant donner le fonctionnement de ce type de jeux
		nous avons découpé sa création de la façon la plus logique possible :
		\begin{itemize}
			\item[La structure du jeux] : partie gérant le stockage du jeux et le respect des règles.
			\item[L'intéligence artificielle] : gestion de l'arbre de déscision permettant de jouer contre l'ordinnateur. 
			\item[L'interface homme-machine] : il sagit de l'interface de communication entre le pc et le joueur.
		\end{itemize}
		
	\subsection{Choix du language}
	
		Afin de concevoir l'ATARI-GO nous avons choisi le language JAVA et ce car il permet de gérer facilement les grosse structure de donner. 
		Il n'est pas tres optimisé au niveau du temps de calcul mais la création de l'abre de déscision étant dans tout les cas trop gros pour etre fait
		compétement nous avons préféré le confort de programmation a une optimisation de calcul qui au finale ne change pas grand chose.

\section{La structure du jeux}

	\subsection{Présentation}
	
		La structure du jeux contient tout les éléments permettant de jouer une partie, il stocke le jeux permet d'y ajouter un piece, de savoir si la parti est fini
		de récupérer toute les information relatives a une partie.
		
	\subsection{Conception}
	
		Le GO est un jeux ayant un fonctionnement tres simple, il est constituer de pion de deux couleur que l'on place sur une grille. Afin de stocker ces éléments
		il va nous falloir créer plusieur classes : 
			\begin{itemize}
				\item[Coordinate] : permet de stocker les coordonners d'un pion et de comparer deux position sur la grille ;
				\item[GroupPawn] : permet de stocker un liste de coordonner adjacente qui ont des parametre commun, la couleur et le nombre de liberter ;
				\item[Color] : classe tres simple permetant de manipuler la couleur des pions;
				\item[GobanStructure] : Classe principale du jeux qui stocke la grille de jeux et les groupes de pièces qui sont dessus, et qui permet de fair evoluer le jeux.
			\end{itemize}

\section{L'intéligence artificielle}

	\subsection{Présentation}
	
		L'intéligence artificielle est principalement constitué de la création d'un arbre de déscision perméttant a l'ordinnateur de choisir le coup le plus judicieux.
	
	\subsection{Conception}

		Comme demander dans le cahier des charges nous allons implémenter l'algorithme Min-Max, ou plutot ca version amélioré l'algorithme Alpha-Beta pour améliorer le temp de calcul.
		
		Il faut en plus lui adjoindre une fonction d'évaluation d'une étape du jeux, malheureusement nous ne sommes pas des grands joueur de GO et nous ne savons donc pas quel parametre
		sont les plus perinants ce qui risque de donner une IA assez faible.
	
\section{L'interface homme-machine}

		Pour l'interface homme-machine nous avons chosi de l'implémenté de façon graphique en utilisant les outils oufert par l'environnement JAVA
		
\clearpage


\section{La structure du jeux}

\section{L'intéligence artificielle}

\section{L'interface homme-machine}

\clearpage







\chapter{Implémetation}

	\section{La structure du jeux}
	
		\subsection{Coordinates}
			
			Simple classe de stockage qui permet de stocker des coordonners sont la forme de deux entiers,
			ainsi que de comparer deux coordonners.
		
		\subsection{Color}
		
			Simple classe avec des valeurs de base permetant de donner une valeur facilement utilisable a 
			la couleur d'un groupe de pièces.
		
		\subsection{GroupPawn}
		
			Classe permettant de gérer un groupe de pièce, il est composé d'une liste de coordonner, de la couleur de se groupe
			et du nombre de liberter de se groupe.
			
			Les méthodes de cette classe permettent la mise à jour d'un groupe lors de l'ajout d'une piece ou de son retrait. Ainsi
			que des méthodes pour savoir si une coordonner se trouve dans un groupe de pièce.
		
		\subsection{GobanStructure}
		
			Classe principale du jeux, chaque action faites lors d'une partie passe par cette classe, elle contient la liste des groupe 
			de pièces blanches, et la liste des groupes de pièces noires, la taille du goban valant 9 de base ainsi que la grille de jeux
			sous la forme d'un tableau à deux dimension, ou chaque case contient une référence vers le groupe de piece qui contient la pièce
			poser à cette position.

	\section{L'intéligence artificielle}

	
	\section{L'interface homme-machine}
	
		Nous avons utilisé les fonctionnalité déja implementé dans l'environnement JAVA ce qui nous à permis de créer une interface graphique simple et fonctionnelle.
	
\clearpage






\chapter{User Manual}

\section{User Interface}

	\subsection{Game Launch}
		When you launch the game, a match versus 
		
		Au lancement du jeux une partie contre l'IA en difficulté moyenne est lance, le joueur débute toujours et joue les pierre noire.
		
	\subsection{Choix des parties}

		L'interface permet de modifier la difficulté de l'IA ou le type de partie via les onglets de selection en haut de la fenetre, la derniere option relance une partie.
		
		Il est aussi possible de configurer le temp de jeux maximal de l'IA en modifiant la valeursituer en bas a droite de la fenetre.
	
	
	\subsection{Fin de partie}
	
		Une fois que l'un des deux joueurs à pris une chaîne la partie est terminer, il suffiet alors de cliquer sur la fenetre pour relancer la même partie. 
		
		
\section{Instalation du jeux}
	
	
	
\clearpage




% Pour finir l'interligne de 1,5
\end{onehalfspace}

%----------------------------------------
% Pour la bibliographie
%----------------------------------------
% Citer tous les ouvrages/références
%\nocite{*}
% Trier par ordre d'apparition
%\bibliographystyle{unsrt}
% Pour le style de la biblio
%\bibliographystyle{plain.bst}
% Ecrire la biblio ici
%\bibliography{biblio}

\printindex

\appendix


\end{document}
