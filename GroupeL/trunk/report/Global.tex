\documentclass[12pt,a4paper,utf8x]{report}
\usepackage [english]{babel}

% Pour pouvoir utiliser 
\usepackage{ucs}
\usepackage[utf8x]{inputenc}

\usepackage{url} % Pour avoir de belles url
\usepackage {geometry}

% Pour mettre du code source
\usepackage {listings}
% Pour pouvoir passer en paysage
\usepackage{lscape}

% Pour pouvoir faire plusieurs colonnes
\usepackage {multicol}
% Pour crééer un index
\usepackage{makeidx}
\makeindex

% Pour gérer les liens interractifs et les signets Acrobat
\usepackage{hyperref}
\hypersetup{
pdftitle={titre de mon document},
pdfauthor={DEJEAN Charles, POTTIER Vincent},
pdfsubject={Sujet du document},
pdfkeywords={les mots clefs},
bookmarks, % Création du signet
pdfstartview=FitH, % Page de la largeur de la fenêtre
colorlinks=true, % Liens en couleur
linkcolor=black, 	
anchorcolor=black, 	
citecolor=black, 	
filecolor=black, 	
menucolor=black,
runcolor=black,
urlcolor=black, 	
frenchlinks=black,
bookmarksnumbered=true, % Signet numéroté
pdfpagemode=UseOutlines, % Montre les bookmarks.
bookmarksopen =true,
}

% Pour afficher la bibliographie, mais pas nottoc (Table of Contents), notlof (List of Figures) ni notlot (List of Tables)
\usepackage[notlof, notlot]{tocbibind}


% Pour les entetes de page
% \usepackage{fancyheadings}
%\pagestyle{fancy}
%\renewcommand{\sectionmark}[1]{\markboth{#1}{}} 
%\renewcommand{\subsectionmark}[1]{\markright{#1}} 

% Pour l'interligne de 1.5
\usepackage {setspace}
% Pour les marges de la page
\geometry{a4paper, top=2.5cm, bottom=3.5cm, left=1.5cm, right=1.5cm, marginparwidth=1.2cm}

\parskip=5pt %% distance entre § (paragraphe)
\sloppy %% respecter toujours la marge de droite 

% Pour les pénalités :
\interfootnotelinepenalty=150 %note de bas de page
\widowpenalty=150 %% veuves et orphelines
\clubpenalty=150 

%Pour la longueur de l'indentation des paragraphes
\setlength{\parindent}{15mm}

%%%% debut macro pour enlever le nom chapitre %%%%
\makeatletter
\def\@makechapterhead#1{%
  \vspace*{50\p@}%
  {\parindent \z@ \raggedright \normalfont
    \interlinepenalty\@M
    \ifnum \c@secnumdepth >\m@ne
        \Huge\bfseries \thechapter\quad
    \fi
    \Huge \bfseries #1\par\nobreak
    \vskip 40\p@
  }}

\def\@makeschapterhead#1{%
  \vspace*{50\p@}%
  {\parindent \z@ \raggedright
    \normalfont
    \interlinepenalty\@M
    \Huge \bfseries  #1\par\nobreak
    \vskip 40\p@
  }}
\makeatother
%%%% fin macro %%%%

%Couverture 
\title
{
	\normalsize{DES Xxxxx xxxx xxxxx\\
	Université de Xxxx Xxxxxx\\
	2004-2005}\\
	\vspace{15mm}
	\Huge{Titre du rapport de stage}
}
\author{DEJEAN Charles, POTTIER Vincent\\
	\vspace{45mm}
}

\date{	
	\normalsize{Lieu du stage\\
	Adresse du stage\\
	Ville du stage\\ 
	\vspace{5mm}	
	Directeur de recherche : M. DUPONT \\
	Rapporteur universitaire : Mme DUPUIS
	}
}

\begin{document}

\maketitle

%Remerciements

Je tiens à remercier :
et on met la liste des personnes que l'on remercie. Toto, tutu, titi. et on met la liste des personnes que l'on remercie. Toto, tutu, titi.et on met la liste des personnes que l'on remercie. Toto, tutu, titi.et on met la liste des personnes que l'on remercie. Toto, tutu, titi.


Et on met la liste des personnes que l'on remercie. Toto, tutu, titi.et on met la liste des personnes que l'on remercie. Toto, tutu, titi.et on met la liste des personnes que l'on remercie. Toto, tutu, titi.et on met la liste des personnes que l'on remercie. Toto, tutu, titi.et on met la liste des personnes que l'on remercie. Toto, tutu, titi.

%\clearpage

\tableofcontents
\clearpage

% Pour avoir un interligne de 1,5
\begin{onehalfspace}

\input{}

\input{}

\input{}

% Pour finir l'interligne de 1,5
\end{onehalfspace}

%----------------------------------------
% Pour la bibliographie
%----------------------------------------
% Citer tous les ouvrages/références
%\nocite{*}
% Trier par ordre d'apparition
%\bibliographystyle{unsrt}
% Pour le style de la biblio
%\bibliographystyle{plain.bst}
% Ecrire la biblio ici
%\bibliography{biblio}

\printindex

\appendix


\end{document}
