\chapter{Manuel d'utilisation}

\section{Les Régles}

\subsection{Déroulement du jeux}

	Le go se joue à deux. Celui qui commence joue avec les pierres noires et l'autre avec les blanches. 
	A tour de rôle, les joueurs posent une pierre de leur couleur sur une intersection inoccupée du goban.

	Lorqu'un joueur supprime la dernière liberté d'une chîne adverse, il la capture et comme nous jouons à
	l'ATARI-GO cela termine la partie.

\subsection{Comment jouer}

	Il est possible de poser une pierres sur n'importe qu'elle intersection de Goban temps qu'elle est libre. Il y
	a cependant une condition le coup ne doit pas être un scuicide. Un suicide consiste a poser un pierre qui n'aura aucune liberté,
	cela est possible si et seulement si se scuicide permet de prendre une chaîne adverse.

\subsection{Petit lexique}

	\begin{itemize}
		\item[Chaîne] : Une chaîne est un ensemble de une ou plusieurs pierres de même couleur voisines de proche en proche. ;
		\item[Libertées] : Les libertés d'une chaîne sont les intersections inoccupées voisines des pierres de cette chaîne.;
	\end{itemize}

	\begin{figure}[10cm]
		\centering
		\includegraphics{chaine_et_lib.bmp}
		\caption{Chaîne et libertées en image}
	\end{figure}
	
\section{L'interface de jeux}

	\subsection{Lancement du jeux}
	
		Au lancement du jeux une partie contre l'IA en difficulté moyenne est lance, le joueur débute toujours et joue les pierre noire.
		
	\subsection{Choix des parties}

		L'interface permet de modifier la difficulté de l'IA ou le type de partie via les onglets de selection en haut de la fenetre, la derniere option relance une partie.
		
		Il est aussi possible de configurer le temp de jeux maximal de l'IA en modifiant la valeursituer en bas a droite de la fenetre.
	
		\begin{figure}[10cm]
			\centering
			\includegraphics{fenetre.bmp}
			\caption{Description de la fenetre de jeux}
		\end{figure}
	
	\subsection{Fin de partie}
	
		Une fois que l'un des deux joueurs à pris une chaîne la partie est terminer, il suffiet alors de cliquer sur la fenetre pour relancer la même partie. 
		
		\begin{figure}[10cm]
			\centering
			\includegraphics{fin.bmp}
			\caption{Description d'une fin de partie}
		\end{figure}
		
\section{Instalation du jeux}
	
	
	
\clearpage
