\chapter{Implémetation}

	\section{La structure du jeux}
	
		\subsection{Coordinates}
			
			Simple classe de stockage qui permet de stocker des coordonners sont la forme de deux entiers,
			ainsi que de comparer deux coordonners.
		
		\subsection{Color}
		
			Simple classe avec des valeurs de base permetant de donner une valeur facilement utilisable a 
			la couleur d'un groupe de pièces.
		
		\subsection{GroupPawn}
		
			Classe permettant de gérer un groupe de pièce, il est composé d'une liste de coordonner, de la couleur de se groupe
			et du nombre de liberter de se groupe.
			
			Les méthodes de cette classe permettent la mise à jour d'un groupe lors de l'ajout d'une piece ou de son retrait. Ainsi
			que des méthodes pour savoir si une coordonner se trouve dans un groupe de pièce.
		
		\subsection{GobanStructure}
		
			Classe principale du jeux, chaque action faites lors d'une partie passe par cette classe, elle contient la liste des groupe 
			de pièces blanches, et la liste des groupes de pièces noires, la taille du goban valant 9 de base ainsi que la grille de jeux
			sous la forme d'un tableau à deux dimension, ou chaque case contient une référence vers le groupe de piece qui contient la pièce
			poser à cette position.

	\section{L'intéligence artificielle}

	
	\section{L'interface homme-machine}
	
		Nous avons utilisé les fonctionnalité déja implementé dans l'environnement JAVA ce qui nous à permis de créer une interface graphique simple et fonctionnelle.
	
\clearpage