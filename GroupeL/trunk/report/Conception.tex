\chapter{Conception}

\section{Choix de conception}

	\subsection{Mise en place}

		Dans un premier temps nous avons cherché à découper le projet en sous parties distinct. Etant donner le fonctionnement de ce type de jeux
		nous avons découpé sa création de la façon la plus logique possible :
		\begin{itemize}
			\item[La structure du jeux] : partie gérant le stockage du jeux et le respect des règles.
			\item[L'intéligence artificielle] : gestion de l'arbre de déscision permettant de jouer contre l'ordinnateur. 
			\item[L'interface homme-machine] : il sagit de l'interface de communication entre le pc et le joueur.
		\end{itemize}
		
	\subsection{Choix du language}
	
		Afin de concevoir l'ATARI-GO nous avons choisi le language JAVA et ce car il permet de gérer facilement les grosse structure de donner. 
		Il n'est pas tres optimisé au niveau du temps de calcul mais la création de l'abre de déscision étant dans tout les cas trop gros pour etre fait
		compétement nous avons préféré le confort de programmation a une optimisation de calcul qui au finale ne change pas grand chose.

\section{La structure du jeux}

	\subsection{Présentation}
	
		La structure du jeux contient tout les éléments permettant de jouer une partie, il stocke le jeux permet d'y ajouter un piece, de savoir si la parti est fini
		de récupérer toute les information relatives a une partie.
		
	\subsection{Conception}
	
		Le GO est un jeux ayant un fonctionnement tres simple, il est constituer de pion de deux couleur que l'on place sur une grille. Afin de stocker ces éléments
		il va nous falloir créer plusieur classes : 
			\begin{itemize}
				\item[Coordinate] : permet de stocker les coordonners d'un pion et de comparer deux position sur la grille ;
				\item[GroupPawn] : permet de stocker un liste de coordonner adjacente qui ont des parametre commun, la couleur et le nombre de liberter ;
				\item[Color] : classe tres simple permetant de manipuler la couleur des pions;
				\item[GobanStructure] : Classe principale du jeux qui stocke la grille de jeux et les groupes de pièces qui sont dessus, et qui permet de fair evoluer le jeux.
			\end{itemize}

\section{L'intéligence artificielle}

	\subsection{Présentation}
	
		L'intéligence artificielle est principalement constitué de la création d'un arbre de déscision perméttant a l'ordinnateur de choisir le coup le plus judicieux.
	
	\subsection{Conception}

		Comme demander dans le cahier des charges nous allons implémenter l'algorithme Min-Max, ou plutot ca version amélioré l'algorithme Alpha-Beta pour améliorer le temp de calcul.
		
		Il faut en plus lui adjoindre une fonction d'évaluation d'une étape du jeux, malheureusement nous ne sommes pas des grands joueur de GO et nous ne savons donc pas quel parametre
		sont les plus perinants ce qui risque de donner une IA assez faible.
	
\section{L'interface homme-machine}

		Pour l'interface homme-machine nous avons chosi de l'implémenté de façon graphique en utilisant les outils oufert par l'environnement JAVA
		
\clearpage


\section{La structure du jeux}

\section{L'intéligence artificielle}

\section{L'interface homme-machine}

	


\subsection{Titre de la sous section}

Ici du texte et du blabla, ce que l'on veut dire et écrire. A remplacer. Ici du texte et du blabla, ce que l'on veut dire et écrire. On peut faire une citation \cite{Motclef1}.
A remplacer. Ici du texte et du blabla, ce que l'on veut dire et écrire. A remplacer. Ici du texte et du blabla, ce que l'on veut dire et écrire. A remplacer. Ici du texte et du blabla, ce que l'on veut dire et écrire. A remplacer. Ici du texte et du blabla, ce que l'on veut dire et écrire. A remplacer.

Ici du texte et du blabla, ce que l'on veut dire et écrire. A remplacer. Ici du texte et du blabla, ce que l'on veut dire et écrire. A remplacer.
Ici du texte et du blabla, ce que l'on veut dire et écrire. A remplacer. Ici du texte et du blabla, ce que l'on veut dire et écrire. A remplacer. Ici du texte et du blabla, ce que l'on veut dire et écrire. A remplacer. Ici du texte et du blabla, ce que l'on veut dire et écrire. A remplacer.

%-- Note de bas de page sur les stades
\protect\footnote{Par exemple, on peut faire un pied de page :
\begin{itemize}
\item avec une liste à puces ;
\item avec une liste à puces ;
\item avec une liste à puces.
\end{itemize}
}
%-- Fin Note de bas de page sur les stades

Ici du texte et du blabla, ce que l'on veut dire et écrire. A remplacer. Ici du texte et du blabla, ce que l'on veut dire et écrire. A remplacer. Ici du texte et du blabla, ce que l'on veut dire et écrire. A remplacer. Ici du texte et du blabla, ce que l'on veut dire et écrire. A remplacer. Ici du texte et du blabla, ce que l'on veut dire et écrire. A remplacer. Ici du texte et du blabla, ce que l'on veut dire et écrire. A remplacer.

\begin{itemize}
\item avec une liste à puces ;
\item avec une liste à puces ;
\item avec une liste à puces.
\end{itemize}

Ici du texte et du blabla, ce que l'on veut dire et écrire. A remplacer. Ici du texte et du blabla, ce que l'on veut dire et écrire. A remplacer. Ici du texte et du blabla, ce que l'on veut dire et écrire. A remplacer. Ici du texte et du blabla, ce que l'on veut dire et écrire. A remplacer. Ici du texte et du blabla, ce que l'on veut dire et écrire. A remplacer. Ici du texte et du blabla, ce que l'on veut dire et écrire. A remplacer.

\subsubsection{Titre de la sous sous section}

Ici du texte et du blabla, ce que l'on veut dire et écrire. A remplacer. Ici du texte et du blabla, ce que l'on veut dire et écrire. A remplacer. Ici du texte et du blabla, ce que l'on veut dire et écrire. A remplacer. Ici du texte et du blabla, ce que l'on veut dire et écrire. A remplacer. Ici du texte et du blabla, ce que l'on veut dire et écrire. A remplacer. Ici du texte et du blabla, ce que l'on veut dire et écrire. A remplacer.

Ici du texte et du blabla, ce que l'on veut dire et écrire. A remplacer. Ici du texte et du blabla, ce que l'on veut dire et écrire. A remplacer. Ici du texte et du blabla, ce que l'on veut dire et écrire. A remplacer. Ici du texte et du blabla, ce que l'on veut dire et écrire. A remplacer. Ici du texte et du blabla, ce que l'on veut dire et écrire. A remplacer. Ici du texte et du blabla, ce que l'on veut dire et écrire. A remplacer.

\subsubsection{Titre de la sous sous section}

Ici du texte et du blabla, ce que l'on veut dire et écrire. A remplacer. Ici du texte et du blabla, ce que l'on veut dire et écrire. A remplacer. Ici du texte et du blabla, ce que l'on veut dire et écrire. A remplacer. Ici du texte et du blabla, ce que l'on veut dire et écrire. A remplacer. Ici du texte et du blabla, ce que l'on veut dire et écrire. A remplacer. Ici du texte et du blabla, ce que l'on veut dire et écrire. A remplacer.

Ici du texte et du blabla, ce que l'on veut dire et écrire. A remplacer. Ici du texte et du blabla, ce que l'on veut dire et écrire. A remplacer. Ici du texte et du blabla, ce que l'on veut dire et écrire. A remplacer. Ici du texte et du blabla, ce que l'on veut dire et écrire. A remplacer. Ici du texte et du blabla, ce que l'on veut dire et écrire. A remplacer. Ici du texte et du blabla, ce que l'on veut dire et écrire. A remplacer.

\subsection{Conclusion}

Ici du texte et du blabla, ce que l'on veut dire et écrire. A remplacer. Ici du texte et du blabla, ce que l'on veut dire et écrire. A remplacer. Ici du texte et du blabla, ce que l'on veut dire et écrire. A remplacer. Ici du texte et du blabla, ce que l'on veut dire et écrire. A remplacer. Ici du texte et du blabla, ce que l'on veut dire et écrire. A remplacer. Ici du texte et du blabla, ce que l'on veut dire et écrire. A remplacer.

Ici du texte et du blabla, ce que l'on veut dire et écrire. A remplacer. Ici du texte et du blabla, ce que l'on veut dire et écrire. A remplacer. Ici du texte et du blabla, ce que l'on veut dire et écrire. A remplacer. Ici du texte et du blabla, ce que l'on veut dire et écrire. A remplacer. Ici du texte et du blabla, ce que l'on veut dire et écrire. A remplacer. Ici du texte et du blabla, ce que l'on veut dire et écrire. A remplacer.

\subsection{Titre de la sous section}

Ici du texte et du blabla, ce que l'on veut dire et écrire. A remplacer. Ici du texte et du blabla, ce que l'on veut dire et écrire. On peut faire une citation \cite{Motclef1}.
A remplacer. Ici du texte et du blabla, ce que l'on veut dire et écrire. A remplacer. Ici du texte et du blabla, ce que l'on veut dire et écrire. A remplacer. Ici du texte et du blabla, ce que l'on veut dire et écrire. A remplacer. Ici du texte et du blabla, ce que l'on veut dire et écrire. A remplacer.

Ici du texte et du blabla, ce que l'on veut dire et écrire. A remplacer. Ici du texte et du blabla, ce que l'on veut dire et écrire. A remplacer.
Ici du texte et du blabla, ce que l'on veut dire et écrire. A remplacer. Ici du texte et du blabla, ce que l'on veut dire et écrire. A remplacer. Ici du texte et du blabla, ce que l'on veut dire et écrire. A remplacer. Ici du texte et du blabla, ce que l'on veut dire et écrire. A remplacer.

\subsection{Titre de la sous section}

Ici du texte et du blabla, ce que l'on veut dire et écrire. A remplacer. Ici du texte et du blabla, ce que l'on veut dire et écrire. On peut faire une citation \cite{Motclef1}.
A remplacer. Ici du texte et du blabla, ce que l'on veut dire et écrire. A remplacer. Ici du texte et du blabla, ce que l'on veut dire et écrire. A remplacer. Ici du texte et du blabla, ce que l'on veut dire et écrire. A remplacer. Ici du texte et du blabla, ce que l'on veut dire et écrire. A remplacer.

Ici du texte et du blabla, ce que l'on veut dire et écrire. A remplacer. Ici du texte et du blabla, ce que l'on veut dire et écrire. A remplacer.
Ici du texte et du blabla, ce que l'on veut dire et écrire. A remplacer. Ici du texte et du blabla, ce que l'on veut dire et écrire. A remplacer. Ici du texte et du blabla, ce que l'on veut dire et écrire. A remplacer. Ici du texte et du blabla, ce que l'on veut dire et écrire. A remplacer.

\clearpage
